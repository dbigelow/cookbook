\documentclass{article}
\usepackage{amsmath,amssymb,verbatim}
\usepackage[right=1in,top=1in,left=1in,bottom=1in]{geometry}
\usepackage{graphicx}
\newcounter{qcounter}
\usepackage{algorithmic}
\usepackage{algorithm}
\begin{document}


\section*{Chicken and Dumpling Stew}



\subsection*{Ingredients}
\subsubsection*{For the Stew}

2 lbs chicken meat cubed (or one large can of meat)\\
1 - 2 cloves garlic, minced\\
1 onion, chopped\\
1 carrot, sliced\\
3 celery tops, chopped small\\
3 whole cloves\\
1 bay leaf\\
1 tablespoon salt\\
water or chicken broth\\

\subsubsection*{For the Dumplings}
$1 \frac{1}{3}$ cups flour\\
2 teaspoons baking powder\\
1 teaspoon parsley flakes or thyme\\
$\frac{1}{2}$ teaspoon salt\\
$\frac{2}{3}$ cup milk\\
2 tablespoons oil\\

\subsection*{Directions}
\begin{list}{\arabic{qcounter}:~}{\usecounter{qcounter}}
\item Cook chicken and garlic in wok with a little oil until outside of chicken is browned. (Not necessary if using canned chicken).
\item Remove chicken meat and set aside.
\item Put onion, carrot, celery, cloves, bay leaf, and salt in wok, with enough water or broth to fully cover the vegetables and chicken combined.
\item Simmer until vegetables are half done.
\item Mix dry ingredients of dumpling mix in a separate bowl.
\item Add chicken back into wok, and bring back to a simmer.
\item Add milk and oil to dry dumpling ingredients, and mix until dough forms.
\item Drop spoonfuls of dumpling mix onto simmering broth.  Cover with lid. Cook for about 10 minutes, or until dumplings are light, fluffy, and done all the way through. (The inside should by fairly dry with lots of air bubbles)
\item Remove from heat and serve.
\end{list}

\end{document}